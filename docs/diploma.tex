%%%%%%%%%%%%%%%%%%%%%%%%%%%%%%%%%%%%%%%%
% datoteka diploma-FRI-vzorec.tex
%
% vzorčna datoteka za pisanje diplomskega dela v formatu LaTeX
% na UL Fakulteti za računalništvo in informatiko
%
% na osnovi starejših verzij vkup spravil Franc Solina, maj 2021
% prvo verzijo je leta 2010 pripravil Gašper Fijavž
%
% za upravljanje z literaturo ta vezija uporablja BibLaTeX
%
% svetujemo uporabo Overleaf.com - na tej spletni implementaciji LaTeXa ta vzorec zagotovo pravilno deluje
%

\documentclass[a4paper,12pt,openright]{book}
%\documentclass[a4paper, 12pt, openright, draft]{book}  Nalogo preverite tudi z opcijo draft, ki pokaže, katere vrstice so predolge! Pozor, v draft opciji, se slike ne pokažejo!
 
\usepackage[utf8]{inputenc}   % omogoča uporabo slovenskih črk kodiranih v formatu UTF-8
\usepackage[slovene,english]{babel}    % naloži, med drugim, slovenske delilne vzorce
\usepackage[pdftex]{graphicx}  % omogoča vlaganje slik različnih formatov
\usepackage{fancyhdr}          % poskrbi, na primer, za glave strani
\usepackage{amssymb}           % dodatni matematični simboli
\usepackage{amsmath}           % eqref, npr.
\usepackage{hyperxmp}
\usepackage[hyphens]{url}
\usepackage{csquotes}
\usepackage[pdftex, colorlinks=true,
						citecolor=black, filecolor=black, 
						linkcolor=black, urlcolor=black,
						pdfproducer={LaTeX}, pdfcreator={LaTeX}]{hyperref}

\usepackage{color}
\usepackage{soul}

\usepackage[
backend=biber,
style=numeric,
sorting=nty,
]{biblatex}


\addbibresource{literatura.bib} %Imports bibliography file


%%%%%%%%%%%%%%%%%%%%%%%%%%%%%%%%%%%%%%%%
%	DIPLOMA INFO
%%%%%%%%%%%%%%%%%%%%%%%%%%%%%%%%%%%%%%%%
\newcommand{\ttitle}{Razvoj vgrajenega podatkovnega sistema}
\newcommand{\ttitleEn}{Diploma thesis template}
\newcommand{\tsubject}{\ttitle}
\newcommand{\tsubjectEn}{\ttitleEn}
\newcommand{\tauthor}{Janez Sedeljšak}
\newcommand{\tkeywords}{Podatkovne baze, C++, Python, B+ drevesa, Podatkovne strukture}
\newcommand{\tkeywordsEn}{Databases, C++, Python, B+ trees, Data structures}

%%%%%%%%%%%%%%%%%%%%%%%%%%%%%%%%%%%%%%%%
%	HYPERREF SETUP
%%%%%%%%%%%%%%%%%%%%%%%%%%%%%%%%%%%%%%%%
\hypersetup{pdftitle={\ttitle}}
\hypersetup{pdfsubject=\ttitleEn}
\hypersetup{pdfauthor={\tauthor}}
\hypersetup{pdfkeywords=\tkeywordsEn}

%%%%%%%%%%%%%%%%%%%%%%%%%%%%%%%%%%%%%%%%
% postavitev strani
%%%%%%%%%%%%%%%%%%%%%%%%%%%%%%%%%%%%%%%%  

\addtolength{\marginparwidth}{-20pt} % robovi za tisk
\addtolength{\oddsidemargin}{40pt}
\addtolength{\evensidemargin}{-40pt}

\renewcommand{\baselinestretch}{1.3} % ustrezen razmik med vrsticami
\setlength{\headheight}{15pt}        % potreben prostor na vrhu
\renewcommand{\chaptermark}[1]%
{\markboth{\MakeUppercase{\thechapter.\ #1}}{}} \renewcommand{\sectionmark}[1]%
{\markright{\MakeUppercase{\thesection.\ #1}}} \renewcommand{\headrulewidth}{0.5pt} \renewcommand{\footrulewidth}{0pt}
\fancyhf{}
\fancyhead[LE,RO]{\sl \thepage} 
%\fancyhead[LO]{\sl \rightmark} \fancyhead[RE]{\sl \leftmark}
\fancyhead[RE]{\sc \tauthor}              % dodal Solina
\fancyhead[LO]{\sc Diplomska naloga}     % dodal Solina


\newcommand{\BibLaTeX}{{\sc Bib}\LaTeX}
\newcommand{\BibTeX}{{\sc Bib}\TeX}

%%%%%%%%%%%%%%%%%%%%%%%%%%%%%%%%%%%%%%%%
% naslovi
%%%%%%%%%%%%%%%%%%%%%%%%%%%%%%%%%%%%%%%%  

\newcommand{\autfont}{\Large}
\newcommand{\titfont}{\LARGE\bf}
\newcommand{\clearemptydoublepage}{\newpage{\pagestyle{empty}\cleardoublepage}}
\setcounter{tocdepth}{1}	      % globina kazala

%%%%%%%%%%%%%%%%%%%%%%%%%%%%%%%%%%%%%%%%
% konstrukti
%%%%%%%%%%%%%%%%%%%%%%%%%%%%%%%%%%%%%%%%  
\newtheorem{izrek}{Izrek}[chapter]
\newtheorem{trditev}{Trditev}[izrek]
\newenvironment{dokaz}{\emph{Dokaz.}\ }{\hspace{\fill}{$\Box$}}


%%%%%%%%%%%%%%%%%%%%%%%%%%%%%%%%%%%%%%%%%%%%%%%%%%%%%%%%%%%%%%%%%%%%%%%%%%%%%%%
%% PDF-A
%%%%%%%%%%%%%%%%%%%%%%%%%%%%%%%%%%%%%%%%%%%%%%%%%%%%%%%%%%%%%%%%%%%%%%%%%%%%%%%

%%%%%%%%%%%%%%%%%%%%%%%%%%%%%%%%%%%%%%%% 
% define medatata
%%%%%%%%%%%%%%%%%%%%%%%%%%%%%%%%%%%%%%%% 
\def\Title{\ttitle}
\def\Author{\tauthor, js0578@student.uni-lj.si}
\def\Subject{\ttitleEn}
\def\Keywords{\tkeywordsEn}

%%%%%%%%%%%%%%%%%%%%%%%%%%%%%%%%%%%%%%%% 
% \convertDate converts D:20080419103507+02'00' to 2008-04-19T10:35:07+02:00
%%%%%%%%%%%%%%%%%%%%%%%%%%%%%%%%%%%%%%%% 
\def\convertDate{%
    \getYear
}

{\catcode`\D=12
 \gdef\getYear D:#1#2#3#4{\edef\xYear{#1#2#3#4}\getMonth}
}
\def\getMonth#1#2{\edef\xMonth{#1#2}\getDay}
\def\getDay#1#2{\edef\xDay{#1#2}\getHour}
\def\getHour#1#2{\edef\xHour{#1#2}\getMin}
\def\getMin#1#2{\edef\xMin{#1#2}\getSec}
\def\getSec#1#2{\edef\xSec{#1#2}\getTZh}
\def\getTZh +#1#2{\edef\xTZh{#1#2}\getTZm}
\def\getTZm '#1#2'{%
    \edef\xTZm{#1#2}%
    \edef\convDate{\xYear-\xMonth-\xDay T\xHour:\xMin:\xSec+\xTZh:\xTZm}%
}

%\expandafter\convertDate\pdfcreationdate 

%%%%%%%%%%%%%%%%%%%%%%%%%%%%%%%%%%%%%%%%
% get pdftex version string
%%%%%%%%%%%%%%%%%%%%%%%%%%%%%%%%%%%%%%%% 
\newcount\countA
\countA=\pdftexversion
\advance \countA by -100
\def\pdftexVersionStr{pdfTeX-1.\the\countA.\pdftexrevision}


%%%%%%%%%%%%%%%%%%%%%%%%%%%%%%%%%%%%%%%%
% XMP data
%%%%%%%%%%%%%%%%%%%%%%%%%%%%%%%%%%%%%%%%  
\usepackage{xmpincl}
%\includexmp{pdfa-1b}

%%%%%%%%%%%%%%%%%%%%%%%%%%%%%%%%%%%%%%%%
% pdfInfo
%%%%%%%%%%%%%%%%%%%%%%%%%%%%%%%%%%%%%%%%  
\pdfinfo{%
    /Title    (\ttitle)
    /Author   (\tauthor, damjan@cvetan.si)
    /Subject  (\ttitleEn)
    /Keywords (\tkeywordsEn)
    /ModDate  (\pdfcreationdate)
    /Trapped  /False
}

%%%%%%%%%%%%%%%%%%%%%%%%%%%%%%%%%%%%%%%%
% znaki za copyright stran
%%%%%%%%%%%%%%%%%%%%%%%%%%%%%%%%%%%%%%%%  

\newcommand{\CcImageCc}[1]{%
	\includegraphics[scale=#1]{cc_cc_30.pdf}%
}
\newcommand{\CcImageBy}[1]{%
	\includegraphics[scale=#1]{cc_by_30.pdf}%
}
\newcommand{\CcImageSa}[1]{%
	\includegraphics[scale=#1]{cc_sa_30.pdf}%
}

%%%%%%%%%%%%%%%%%%%%%%%%%%%%%%%%%%%%%%%%%%%%%%%%%%%%%%%%%%%%%%%%%%%%%%%%%%%%%%%
%%%%%%%%%%%%%%%%%%%%%%%%%%%%%%%%%%%%%%%%%%%%%%%%%%%%%%%%%%%%%%%%%%%%%%%%%%%%%%%

\begin{document}
\selectlanguage{slovene}
\frontmatter
\setcounter{page}{1} %
\renewcommand{\thepage}{}       % preprečimo težave s številkami strani v kazalu

%%%%%%%%%%%%%%%%%%%%%%%%%%%%%%%%%%%%%%%%
%naslovnica
 \thispagestyle{empty}%
   \begin{center}
    {\large\sc Univerza v Ljubljani\\%
%      Fakulteta za elektrotehniko\\% za študijski program Multimedija
%      Fakulteta za upravo\\% za študijski program Upravna informatika
      Fakulteta za računalništvo in informatiko\\%
%      Fakulteta za matematiko in fiziko\\% za študijski program Računalništvo in matematika
     }
    \vskip 10em%
    {\autfont \tauthor\par}%
    {\titfont \ttitle \par}%
    {\vskip 3em \textsc{DIPLOMSKO DELO\\[5mm]         % dodal Solina za ostale študijske programe
    VISOKOŠOLSKI STROKOVNI ŠTUDIJSKI PROGRAM\\ PRVE STOPNJE\\ RAČUNALNIŠTVO IN INFORMATIKA}\par}%
%     UNIVERZITETNI  ŠTUDIJSKI PROGRAM\\ PRVE STOPNJE\\ RAČUNALNIŠTVO IN INFORMATIKA}\par}%
%    INTERDISCIPLINARNI UNIVERZITETNI\\ ŠTUDIJSKI PROGRAM PRVE STOPNJE\\ MULTIMEDIJA}\par}%
%    INTERDISCIPLINARNI UNIVERZITETNI\\ ŠTUDIJSKI PROGRAM PRVE STOPNJE\\ UPRAVNA INFORMATIKA}\par}%
%    INTERDISCIPLINARNI UNIVERZITETNI\\ ŠTUDIJSKI PROGRAM PRVE STOPNJE\\ RAČUNALNIŠTVO IN MATEMATIKA}\par}%
    \vfill\null%
% izberite pravi habilitacijski naziv mentorja!
    {\large \textsc{Mentor}: doc. dr. Boštjan Slivnik\par}%
    {\vskip 2em \large Ljubljana, \the\year \par}%
\end{center}
% prazna stran
%\clearemptydoublepage      
% izjava o licencah itd. se izpiše na hrbtni strani naslovnice

%%%%%%%%%%%%%%%%%%%%%%%%%%%%%%%%%%%%%%%%
%copyright stran
%%%%%%%%%%%%%%%%%%%%%%%%%%%%%%%%%%%%%%%%
\newpage
\thispagestyle{empty}

\vspace*{5cm}
{\small \noindent
To delo je ponujeno pod licenco \textit{Creative Commons Priznanje avtorstva-Deljenje pod enakimi pogoji 2.5 Slovenija} (ali novej\v so razli\v cico).
To pomeni, da se tako besedilo, slike, grafi in druge sestavine dela kot tudi rezultati diplomskega dela lahko prosto distribuirajo,
reproducirajo, uporabljajo, priobčujejo javnosti in predelujejo, pod pogojem, da se jasno in vidno navede avtorja in naslov tega
dela in da se v primeru spremembe, preoblikovanja ali uporabe tega dela v svojem delu, lahko distribuira predelava le pod
licenco, ki je enaka tej.
Podrobnosti licence so dostopne na spletni strani \href{http://creativecommons.si}{creativecommons.si} ali na Inštitutu za
intelektualno lastnino, Streliška 1, 1000 Ljubljana.

\vspace*{1cm}
\begin{center}% 0.66 / 0.89 = 0.741573033707865
%{ \CcImageCc{0.741573033707865}\hspace*{1ex}\CcImageBy{1}\hspace*{1ex}\CcImageSa{1}% }%
\end{center}
}

\vspace*{1cm}
{\small \noindent
Izvorna koda diplomskega dela, njeni rezultati in v ta namen razvita programska oprema je ponujena pod licenco GNU General Public License,
različica 3 (ali novejša). To pomeni, da se lahko prosto distribuira in/ali predeluje pod njenimi pogoji.
Podrobnosti licence so dostopne na spletni strani \url{http://www.gnu.org/licenses/}.
}

\vfill
\begin{center} 
\ \\ \vfill
{\em
Besedilo je oblikovano z urejevalnikom besedil \LaTeX.}
\end{center}

% prazna stran
\clearemptydoublepage

%%%%%%%%%%%%%%%%%%%%%%%%%%%%%%%%%%%%%%%%
% stran 3 med uvodnimi listi
\thispagestyle{empty}
\
\vfill

\bigskip
\noindent\textbf{Kandidat:} Janez Sedeljšak\\
\noindent\textbf{Naslov:} Razvoj vgrajenega podatkovnega sistema\\
\noindent\textbf{Vrsta naloge:} Diplomska naloga na visokošolskem programu prve stopnje Računalništvo in informatika \\
\noindent\textbf{Mentor:} doc. dr. Boštjan Slivnik\\

\bigskip
\noindent\textbf{Opis:}\\
Besedilo teme diplomskega dela študent prepiše iz študijskega informacijskega sistema, kamor ga je vnesel mentor. 
V nekaj stavkih bo opisal, kaj pričakuje od kandidatovega diplomskega dela. 
Kaj so cilji, kakšne metode naj uporabi, morda bo zapisal tudi ključno literaturo.

\bigskip
\noindent\textbf{Title:} Development of an embeded database system

\bigskip
\noindent\textbf{Description:}\\
opis diplome v angleščini

\vfill



\vspace{2cm}

% prazna stran
\clearemptydoublepage

% zahvala
%\thispagestyle{empty}\mbox{}\vfill\null\it%
%\noindent
%Na tem mestu zapišite, komu se zahvaljujete za pomoč pri izdelavi diplomske naloge oziroma pri vašem študiju %nasploh. Pazite, da ne boste koga pozabili. Utegnil vam bo zameriti. Temu se da izogniti tako, da celotno %zahvalo izpustite.
%$\rm\normalfont

% prazna stran
\clearemptydoublepage

%%%%%%%%%%%%%%%%%%%%%%%%%%%%%%%%%%%%%%%%
% posvetilo, če sama zahvala ne zadošča :-)
%\thispagestyle{empty}\mbox{}{\vskip0.20\textheight}\mbox{}\hfill\begin{minipage}{0.55\textwidth}%
%Svoji dragi Alenčici.
%\normalfont\end{minipage}

% prazna stran
\clearemptydoublepage


%%%%%%%%%%%%%%%%%%%%%%%%%%%%%%%%%%%%%%%%
% kazalo
\setcounter{tocdepth}{3}
\pagestyle{empty}
\def\thepage{}% preprečimo težave s številkami strani v kazalu
\tableofcontents{}


% prazna stran
\clearemptydoublepage

%%%%%%%%%%%%%%%%%%%%%%%%%%%%%%%%%%%%%%%%
% seznam kratic

\chapter*{Seznam uporabljenih kratic}

\noindent\begin{tabular}{p{0.11\textwidth}|p{.39\textwidth}|p{.39\textwidth}}    % po potrebi razširi prvo kolono tabele na račun drugih dveh!
  {\bf kratica} & {\bf angleško}                              & {\bf slovensko} \\ \hline
  {\bf DBMS} & database management system & sistem za upravljanje podatkovnih baz \\
  {\bf API} & application programming interface & aplikacijski programski vmesnik \\
  {\bf SQL} & structured query language & strukturiran jezik poizvedb \\
  {\bf NoSQL} & ne relacijske podatkovne baze & non relational databases \\
  {\bf I/O} & input/output operations & vhodno/izhodne operacije \\
%  \dots & \dots & \dots \\
\end{tabular}


% prazna stran
\clearemptydoublepage

%%%%%%%%%%%%%%%%%%%%%%%%%%%%%%%%%%%%%%%%
% povzetek
\phantomsection
\addcontentsline{toc}{chapter}{Povzetek}
\chapter*{Povzetek}

\noindent\textbf{Naslov:} \ttitle
\bigskip

\noindent\textbf{Avtor:} \tauthor
\bigskip

%\noindent\textbf{Povzetek:} 
\noindent V diplomskem delu je predstavljenih trenutno nekaj najbolj uporabljenih sistemov za uporabljanje z podatkov (DMBS). V veliki meri so standard podatkovnih baz še vedno relacijske podatkovne baze. V ta namen je tekom dela predstavljen razvoj vgrajenega relacijskega sistema za programski jezik Python.

Sam razvoj namenske knjižnice je pripravljen v programskem jeziku C++, saj gre za nizko nivosjki jezik, kjer imamo visoko fleksibilnost pri upravljanju s pomnilniku. Predstavljen je razvoj vseh potrebnih segmentov za dobro delujočo relacijsko podatkovno bazo. Ključnega pomena tekom razvoja je bila uporaba dobrih podatkovnih struktur in algoritmov, ki dobro izkoristijo I/O operacije, ki jih ponuja operacijski sistem in posledično pripeljejo do dobro pripravljenega podatkovnega sistema.

V zadnjem sklopu diplomskega dela smo pripravili analizo uspešnosti implementacije podatkovnega sistema na različnih scenarijih in ob različnih konfiguracijah. Poleg tega je pripravljena tudi analiza z že obstoječimi DMBS - SQLite in MySQL) ob enakovrednih scenarijih testiranja novo pripravljene knjižnice.
\bigskip

\noindent\textbf{Ključne besede:} \tkeywords.
% prazna stran
\clearemptydoublepage

%%%%%%%%%%%%%%%%%%%%%%%%%%%%%%%%%%%%%%%%
% abstract
\phantomsection
\selectlanguage{english}
\addcontentsline{toc}{chapter}{Abstract}
\chapter*{Abstract}

\noindent\textbf{Title:} \ttitleEn
\bigskip

\noindent\textbf{Author:} \tauthor
\bigskip

%\noindent\textbf{Abstract:} 
\noindent The thesis presents several currently used systems for working with data (DBMS). Relational databases are still widely used as the standard data storage systems. In this context, the development of an embedded relational database system for the Python programming language is presented.

The development of the dedicated library is implemented in the C++ programming language, which is a low-level language providing high flexibility in memory management. The development of all necessary components for a well-functioning relational database is described. During the development process, a key focus was on utilizing efficient data structures and algorithms that make effective use of I/O operations offered by the operating system, resulting in a well-prepared data system.

In the final part of the thesis, a performance analysis of the implemented data system is conducted under different scenarios and configurations. Additionally, an analysis is performed comparing the newly developed library with existing DBMS (SQLite and MySQL) using equivalent testing scenarios.
\bigskip

\noindent\textbf{Keywords:} \tkeywordsEn.
\selectlanguage{slovene}
% prazna stran
\clearemptydoublepage

%%%%%%%%%%%%%%%%%%%%%%%%%%%%%%%%%%%%%%%%
\mainmatter
\setcounter{page}{1}
\pagestyle{fancy}

% združimo uvod in kaj so relacijske podatkovne baze
% 1. začnemo z brez podatkovnih baz dan danes negre (gre za trajen način shranjevanja podatkov itd.)
% 2. različni tipi podatkovnih baz
% 2.1 relacijske
% 2.2 nosql
% 3. relacijske in depth

\chapter{Uvod}
    Živimo v obdobju, kjer velepodatkov. Gre za ogromne količine podatkov, ki so shranjeni na različnih strežniških storitvah. Kadar gre za trajno shranjevanje podatkov govorimo o podatkovnih bazah. Trenutno se omenjeno področje deli na dve večji skupini - relacijske in neralcijske podatkovne baze.
    \section{Različni tipi podatkovnih baz}
        \subsection{Relacijske podatkovne baze}
        Trenutno so standard na trgu še vedno relacijske podatkovne baze. Gre za striktno strukturo entitet, ki vsebujejo smiselne povezave - realcije s pomočjo tako imenovanih tujih ključev. Gre za standard, ki se je prvič pojavil leta 1970, ko ga je razvil IBM \cite{IBM_DMBS_1970}. Razvita je bila prva družina relacijskih podatkovnih baz DB2, katero je razvil Edgar F. Codd - matematik izobražen na univerzi Oxford.
        \subsection{Nerelacijske podatkovne baze}
        Gre za novo skupino podatkovnih baz, ki temeljijo na čist drugačni osnovi kot realcijske podatkovne baze. Pojavile so se kot odgovor na težave, s katerimi se srečujemo pri relacijskih podatkovnih bazah. Kot je zapisal viš. pred. Aljaž Zrnec "Podatkovne baze NoSQL niso bile razvite z namenom popolne zamenjave relacijskih baz" \cite{zrnec2011podatkovne}. Glavna težava pri realcijskih podatkovnih bazah je striktna struktura, ki se je moramo držati. V novi skupini podatkovnih baz (NoSQL) je prioriteta fleksibilnost. Sama struktura podatkov je bistveno drugačne, saj entitete in relacije med zapisi zamenjajo objekti in dedovanje. Vse skupaj prinese še eno veliko prednost, ki jo imajo nerelacijske podatkovne baze - zaradi enkapsulacije posameznih zapisov lahko celotno bazo porazdelimo na več računalnikov (tako imenovana horizontalna skalibilnost, ki je relacijske podatkovne baze ne podpirajo).
    
    \section{Kje uporabljamo relacijske podatkovne baze in kako delujejo?}

    Relacijske podatkovne baze se uporabljajo povsod, kjer imamo velike količine strukturiranih podatkov, ki jih želimo obdržati za trajno shranjevanje. Preprost primer relacijske podatkovne baze:
    
    \begin{figure}[h]
        \centerline{\includegraphics[height=0.5\textwidth, angle=0]{what-is-a-relational-database.jpg}}
        \caption{Preprost primer realcijske podatkovne baze.}
        \label{sl:mindmap}
    \end{figure}

    Na primeru je prikazana struktura, kjer shranjujemo zapise študentov in predmetov, ki se jih udeležujejo. Vsak zapis v relacijski podatkovni bazi ima svoj primarni ključ, preko katerega potekajo vse relacije znotraj podatkovne baze. Entiteti "Students" in "Courses" sta starševski entiteti. "StudentCourses", pa predstavlja povezovalno entiteto med prej omenjenima. Namreč vsak študent je lahko v več predmetih in enako velja za predmete (lahko vključujejo več študentov).

    \section{Motivacija za razvoj lastnega relacijskega sistema}
    Koncept shranjevanja podatkov je v osnovi dokaj preprost. Različni sistemi za shranjevanje podatkov bodisi relacijski ali nerelacijski sistemi za shranjevanje podatkov s seboj prinesejo ogromno abstrakcije.

    Uporabniki podatkovnih baz se zares začnejo zavedati težav, ko sistem za shranjevanje podatkov ni več odziven kot bi si želeli. Tekom razvoja lastnih aplikaciji in preprostih API-jev, se le redko posvetimo optimalnemu delovanju naše podatkovne baze. Težave se začnejo pojavljati, kadar v posamezni entiteti pride do velike količine podatkov in poizvedbe nad podatki in samo iskanje posameznih zapisov postane zakasnjeno.

    Na tej točki so potrebne optimizacije same strukture naše podatkovne baze. Eden najpomembnejših konceptov za hitro iskanje po posameznih atributov je postavitev indeksov. Indeksiranje podatkov je koncept, ki se pojavlja povsod v računalništvu in ne le v relacijskih podatkovnih bazah. Gre za pripravo iskalne strukture, ki nam omogoča bistveno hitrejše iskanje podatkov s pomočjo dobre podatkovne strukture. V praksi se izkaže, da sta nekako najbolj pogosta pristopa indeksiranje z zgoščevalnimi strukturami in drevesnimi strukturami. V smislu relacijskih podatkovnih baz so najpogosteje uporablja več nivojsko indeksiranje, ki je realizirano prav z drevesi (v večini primerov gre za B drevesa). Gre za skupino dreves, kjer ima vsako vozlišče lahko `M` zapisov in `M+1` kazalcev na nova vozlišča. S pomočjo postavitve indeksov lahko linearno iskanje skozi zapise spremenimo v binarno iskanje (oz. iz aproksimacijske notacije O(N) v O(log(N)).

\chapter{Razvoj jedra podatkovnega sistema}
\label{ch0}
    V poglavju predstavimo razvoj jedra za relacijski podatkovni sistem. V prvi fazi je predstavljena celotna strukturiranje podatkov in način shranjevanja na disk. Za tem predstavimo razvoj B+ drevesne strukture za optimalno indeksiranje podatkov. Nato, pa je predsatvljena še optimizacija posameznih segmentov in sama strukture kode na nivoju jedra podatkovnega sistema.
    
    \section{Osnovna hierarhična struktura shranjevanja podatkov}
        \subsection{Tipi podatkov}
        \subsection{Podatki in vezani kazalci}
        \subsection{Struktura relaciji med entitetami}
    \section{Indeksiranje z uporabo B+ dreves}
        \subsection{B+ drevesa in njihove značilnosti}
        \subsection{Implementacija B+ dreves za shranjevanje na disk}
        \subsection{Uporaba B+ dreves v bazi podatkov (iskanje po indeksiranih podatkih)}
        \subsection{Dinamično nalaganje in ohranjanje posameznih segmentov drevesa v pomnilnik}
        \subsection{Slabosti uporabe indeksov}
    \section{Pomembni gradniki za optimalno izvedbo poivedb}
        \subsection{Gradnja dreves za pogojni del poizvedb}
        \subsection{Izvedba poizvedb na večih entitetah hkrati}
        \subsection{Optimizacija poizvedb z gručenjem in primerjava različnih pristopov}
        \subsection{Optimizacija na nivoju urejanja podatkov ob poizvedbah}
    \section{Struktura kode na nivoju jedra}
   % \section{(?) Transakcije in sistem za obnovitev podatkov}
   % section{(?) Migracije znotraj entitet}
   
\chapter{Razvoj knjižnice za delo s podatkovnim sistemom}
\label{ch1}
   \section{Komunikacija med prog. jezikoma C++ in Python}
   \section{Stukture za delo z jedrom in kodna struktura}
   \section{Abstraktna raven v prog. jeziku Python z dobro razvijalsko izkušnjo}

\chapter{Analiza}
\label{ch2}
   \section{Razvoj z uporabo testno usmerjenega pristopa}
   \section{Meritve na različnih napravah za shranjevanje in konfiguracijah}
   \section{Primerjava z ostalimi relacijskimi sistemi}
   \section{Uporaba ORM-jev v Pythonu}

\chapter{Scenariji oz. primeri uporabe}
\label{ch3}
    \section{Strežniško logiranje podatkov}
    \section{Manjši strežniški API}
    \section{Preprosta mobilna aplikacija}

\chapter{Sklepne ugotovitve}


%\cleardoublepage
% \addcontentsline{toc}{chapter}{Literatura}

% če imaš težave poravnati desni rob bibliografije, potem odkomentiraj spodnjo vrstico
\raggedright

% v zadnji verziji diplomskega dela običajno združiš vse tri vrste referenc v en sam seznam in
% izpustiš delne sezname
\printbibliography[heading=bibintoc,title={Literatura}]

\end{document}